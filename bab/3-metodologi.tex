\chapter{METODOLOGI}
\label{chap:desainimplementasi}

% Ubah bagian-bagian berikut dengan isi dari desain dan implementasi

% Penelitian ini dilaksanakan sesuai \lipsum[1][1-5]

\section{Data dan Peralatan / Data dan Alat Bantu / Material}
\label{sec:perlengkapan}

\subsection{Dataset}
Untuk mencapai hasil training model, akan digunakan dataset Safety Helmet Detection yang merupakan dataset berisi 5000 gambar personil proyek yang menggunakan helm proyek dan yang tidak. Pada dataset ini sudah diberikan label pada annotation untuk tiap gambarnya.

\section{Implementasi Alat
\label{sec:implementasi alat}}

Alat diimplementasikan dengan \lipsum[1]

% Contoh pembuatan potongan kode
\begin{lstlisting}[
  language=C++,
  caption={Program halo dunia.},
  label={lst:halodunia}
]
#include <iostream>

int main() {
    std::cout << "Halo Dunia!";
    return 0;
}
\end{lstlisting}

\lipsum[2-3]

% Contoh input potongan kode dari file
\lstinputlisting[
  language=Python,
  caption={Program perhitungan bilangan prima.},
  label={lst:bilanganprima}
]{program/bilangan-prima.py}

\lipsum[4]
